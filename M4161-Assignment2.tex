\documentclass[11pt]{amsart}
\usepackage[utf8]{inputenc}
\usepackage[margin=1in]{geometry}
\usepackage{amsfonts, amssymb, amsmath, amsthm, booktabs, hyperref, pgfplots, tikz, xcolor}

\theoremstyle{definition}\newtheorem{question}{Question}
\theoremstyle{definition}\newtheorem*{solution}{Solution}

\begin{document}

\noindent \textbf{MATH 4161 Mathematics of Cryptography} \hfill \textbf{Assignment 2} \\
\textsc{Assignment} \hfill \textsc{Joe Tran}

\begin{question}
    Given the ciphertext \textsf{UWHRA DRVBI VPQVG MHGBY OBBTP ZLMJO}, and given that
    \begin{enumerate}
        \item The length of the key is \textsf{4}.
        \item The 13th letter is \textsf{U}.
        \item The 22nd letter is \textsf{T}.
        \item The 24th letter is \textsf{P}.
    \end{enumerate}
    Decipher the text into plaintext, and determine the keyword.
\end{question}

\begin{solution}
    Because the length of the key is 4, then we will rewrite the ciphertext as
    \begin{center}
        \textsf{UWHR ADRV BIVP QVGM HGBY OBBT PZLM JO}
    \end{center}
    Because the 13th letter is $U$, then we know that $Q \mapsto U$, so by the Vigenere square, this means that $Q + W = U$. So $W$ is part of the keyword. Then, we have the row of letters corresponding to row $W$ given by
    \begin{center} \footnotesize
        \begin{tabular}{c|cccccccccccccccccccccccccc}
            & A & B & C & D & E & F & G & H & I & J & K & L & M & N & O & P & Q & R & S & T & U & V & W & X & Y & Z \\ \hline
            W & W & X & Y & Z & A & B & C & D & E & F & G & H & I & J & K & L & M & N & O & P & Q & R & S & T & U & V \\
        \end{tabular}
    \end{center}
    and as we see from the table, $Q \mapsto U$, which corresponds to row $W$. This also means that, $W$ is our first letter of the keyword.

    Next, because the 22nd letter is $T$, then we know that $B \mapsto T$, so by the Vigenere square, this means that $B + I = T$, so $I$ is part of the keyword. Then we have the row of letters corresponding to row $I$ as follows
    \begin{center} \footnotesize
        \begin{tabular}{c|cccccccccccccccccccccccccc}
            & A & B & C & D & E & F & G & H & I & J & K & L & M & N & O & P & Q & R & S & T & U & V & W & X & Y & Z \\ \hline
            I & I & J & K & L & M & N & O & P & Q & R & S & T & U & V & W & X & Y & Z & A & B & C & D & E & F & G & H \\
        \end{tabular}
    \end{center}
    as we see from the table, $B \mapsto T$, which corresponds to row $I$. This also means that $I$ is our second letter of the keyword.

    Next, since the 24th letter is $P$, then we know that $T \mapsto P$, so by the Vigenere square, this means that $T + E = P$. So $P$ is part of the keyword. Then we have the row of letters corresponding to row $E$ as follows
    \begin{center} \footnotesize
        \begin{tabular}{c|cccccccccccccccccccccccccc}
            & A & B & C & D & E & F & G & H & I & J & K & L & M & N & O & P & Q & R & S & T & U & V & W & X & Y & Z \\ \hline
            E & E & F & G & H & I & J & K & L & M & N & O & P & Q & R & S & T & U & V & W & X & Y & Z & A & B & C & D \\
        \end{tabular}
    \end{center}
    and as we see from the table, $T \mapsto P$, which corresponds to row $E$. Furthermore, $E$ is the last letter of our keyword.
    
    So our key up to this point is $WIxE$, where $x$ is the missing letter we require. Now with the given tables above, let us transform the ciphertext to what we can so far.
    \begin{center} \tiny
        \begin{tabular}{cccc|cccc|cccc|cccc|cccc|cccc|cccc|cc}
            U & W & H & R & A & D & R & V & B & I & V & P & Q & V & G & M & H & G & B & Y & O & B & B & T & P & Z & L & M & J & O \\ \hline
            Y & O & . & N & E & V & . & R & F & A & . & L & U & N & . & I & L & Y & . & U & S & T & . & P & T & R & . & I & N & G \\
        \end{tabular}
    \end{center}
    
    At this stage, we can start to see the string that is appearing from the keyword that we have formed. We now conjecture that the third letter is $U$, meaning that $H \mapsto U$, so by the Vigenere sqaure, this means that $H + N = U$, so $N$ is part of the keyword. Then we have the row of letters corresponding to row $N$ as follows
    \begin{center} \footnotesize
        \begin{tabular}{c|cccccccccccccccccccccccccc}
            & A & B & C & D & E & F & G & H & I & J & K & L & M & N & O & P & Q & R & S & T & U & V & W & X & Y & Z \\ \hline
            N & N & O & P & Q & R & S & T & U & V & W & X & Y & Z & A & B & C & D & E & F & G & H & I & J & K & L & M \\
        \end{tabular}
    \end{center}

    Finishing the above table,
    \begin{center} \tiny
        \begin{tabular}{cccc|cccc|cccc|cccc|cccc|cccc|cccc|cc}
            U & W & H & R & A & D & R & V & B & I & V & P & Q & V & G & M & H & G & B & Y & O & B & B & T & P & Z & L & M & J & O \\ \hline
            Y & O & U & N & E & V & E & R & F & A & I & L & U & N & T & I & L & Y & O & U & S & T & O & P & T & R & Y & I & N & G \\
        \end{tabular}
    \end{center}

    Therefore, our plaintext is \textsf{YOUN EVER FAIL UNTI LYOU STOP TRYI NG} and our key is \textsf{WINE}.
\end{solution}

\end{document}