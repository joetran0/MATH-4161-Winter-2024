\documentclass[11pt]{article}
\usepackage[utf8]{inputenc}
\usepackage[margin=1in]{geometry}
\usepackage{amsfonts, amssymb, amsmath, amsthm, booktabs, hyperref, pgfplots, tikz, xcolor}

\theoremstyle{definition}\newtheorem{definition}{Definition}
\theoremstyle{definition}\newtheorem{example}{Example}
\theoremstyle{definition}\newtheorem{samplecode}{Sample Code}

\begin{document}

\noindent \textbf{MATH 4161 Mathematics of Cryptography} \hfill \textbf{Lecture 5} \\
\noindent \textsc{Lecture} \hfill \textsc{Joe Tran}

\section{Agenda}

\begin{itemize}
    \item Assignments: 9 will be due on Jan 31.
    \item Hill and Modular Arithmetic
    \item Feistel and Bit Operations
\end{itemize}

\section{Hill and Modular Arithmetic}

The key is a $k \times k$ matrix
\begin{equation*}
    A = \begin{bmatrix}
        a_{11} & a_{12} & \cdots & a_{1k} \\
        a_{21} & a_{22} & \cdots & a_{2k} \\
        \vdots & \vdots & & \vdots \\
        a_{k1} & a_{k2} & \cdots & a_{kk}
    \end{bmatrix}
\end{equation*}
We require that $\gcd(\det(A), 29) = 1$. Given a sequence of values $p_1, p_2,..., p_k \to c_1, c_2,..., c_k$, where for each $i$,
\begin{equation*}
    c_i = a_{i1}p_1 + a_{i2}p_2 + \cdots + a_{ik}p_k \pmod {29}
\end{equation*}
or
\begin{equation*}
    \begin{bmatrix}
        a_{11} & a_{12} & \cdots & a_{1k} \\
        a_{21} & a_{22} & \cdots & a_{2k} \\
        \vdots & \vdots & & \vdots \\
        a_{k1} & a_{k2} & \cdots & a_{kk}
    \end{bmatrix}\begin{bmatrix} p_1 \\ p_2 \\ \vdots \\ p_k \end{bmatrix} = \begin{bmatrix} c_1 \\ c_2 \\ \vdots \\ c_k \end{bmatrix} \pmod {29}
\end{equation*}
then
\begin{equation*}
    \begin{bmatrix} p_1 \\ p_2 \\ \vdots \\ p_k \end{bmatrix} = \begin{bmatrix}
        a_{11} & a_{12} & \cdots & a_{1k} \\
        a_{21} & a_{22} & \cdots & a_{2k} \\
        \vdots & \vdots & & \vdots \\
        a_{k1} & a_{k2} & \cdots & a_{kk}
    \end{bmatrix}^{-1}\begin{bmatrix} c_1 \\ c_2 \\ \vdots \\ c_k \end{bmatrix} \pmod {29}
\end{equation*}

\begin{example}
    If the key $A = \begin{bmatrix} 5 & 13 \\ 1 & 17 \end{bmatrix}$ and the plaintext is GIFT, or 6-8-5-19 then
    \begin{equation*}
        \begin{bmatrix} 5 & 13 \\ 1 & 17 \end{bmatrix} \begin{bmatrix} 6 \\ 8 \end{bmatrix} = \begin{bmatrix} 5 \cdot_{29} 6 +_{29} 8 \cdot_{29} 13 \\ 1 \cdot_{29} 6 +_{29} 17 \cdot_{29} 8\end{bmatrix} = \begin{bmatrix} 18 \\ 26 \end{bmatrix} = \begin{bmatrix} S \\ . \end{bmatrix}
    \end{equation*}
    and
    \begin{equation*}
        \begin{bmatrix} 5 & 13 \\ 1 & 17 \end{bmatrix} \begin{bmatrix} 5 \\ 19 \end{bmatrix} = \begin{bmatrix} 5 \cdot_{29} 5 +_{29} 13 \cdot_{29} 19 \\ 1 \cdot_{29} 5 +_{29} 17 \cdot_{29} 19 \end{bmatrix} = \begin{bmatrix} 11 \\ 9 \end{bmatrix} = \begin{bmatrix}  \end{bmatrix}
    \end{equation*}
    or alternatively,
    \begin{equation*}
        \begin{bmatrix} 5 & 13 \\ 1 & 17 \end{bmatrix}\begin{bmatrix} 6 & 5 \\ 8 & 19 \end{bmatrix} = \begin{bmatrix} 18 & 11 \\ 26 & 9 \end{bmatrix}
    \end{equation*}
\end{example}

\begin{example}
    Recover the plaintext from the ciphertext FVGLNPCJSG given that it was enrypted using the Hill encipherment system with a $2 \times 2$ matrix mod 29 and the plaintext begins with the letters tell.
\end{example}

We seek a key of the form $k = \begin{bmatrix} a & b \\ c & d \end{bmatrix}$. Our plaintext is tell which translates to 19 4 11 11, and so in ciphertext we have FVGL which translates to 5 21 6 11. Then we have
\begin{equation*}
    \begin{bmatrix} a & b \\ c & d \end{bmatrix}\begin{bmatrix} 19 \\ 4 \end{bmatrix} = \begin{bmatrix} 5 \\ 21 \end{bmatrix}
\end{equation*}
and
\begin{equation*}
    \begin{bmatrix} a & b \\ c & d \end{bmatrix}\begin{bmatrix} 11 \\ 11 \end{bmatrix} = \begin{bmatrix} 6 \\ 11 \end{bmatrix}
\end{equation*}
Or in other words,
\begin{align*}
    \begin{bmatrix}
        a & b \\ c & d
    \end{bmatrix}\begin{bmatrix}
        19 & 11 \\ 4 & 11
    \end{bmatrix}\begin{bmatrix}
        19 & 11 \\ 4 & 11
    \end{bmatrix}^{-1} &= \begin{bmatrix} 
        5 & 6 \\ 21 & 11
    \end{bmatrix} \begin{bmatrix}
        19 & 11 \\ 4 & 11
    \end{bmatrix}^{-1} \\
    \begin{bmatrix}
        a & b \\ c & d \end{bmatrix} &= \begin{bmatrix} 
            5 & 6 \\ 21 & 11
        \end{bmatrix} \begin{bmatrix}
            19 & 11 \\ 4 & 11
        \end{bmatrix}^{-1} \\
    \begin{bmatrix}
        a & b \\ c & d 
    \end{bmatrix}^{-1} &= \cdots
\end{align*}
But then
\begin{equation*}
    \begin{bmatrix}
        a & b \\ c & d 
    \end{bmatrix}^{-1h}\begin{bmatrix} 5 & 6 & 13 & 2 & 18 \\ 21 & 11 & 15 & 9 & 6 \end{bmatrix} = \cdots
\end{equation*}


\end{document}