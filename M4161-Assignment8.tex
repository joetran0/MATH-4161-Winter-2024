\documentclass[11pt]{article}
\usepackage[utf8]{inputenc}
\usepackage[dvipsnames]{xcolor}
\usepackage[margin=1in]{geometry}
\usepackage{amsfonts, amssymb, amsmath, amsthm, booktabs, hyperref, pgfplots, tikz, xcolor, mathrsfs}

\theoremstyle{definition}\newtheorem{definition}{Definition}
\theoremstyle{definition}\newtheorem{question}{Question}
\theoremstyle{definition}\newtheorem*{solution}{Solution}
\theoremstyle{definition}\newtheorem{example}{Example}
\theoremstyle{definition}\newtheorem{notation}{Notation}
\theoremstyle{theorem}\newtheorem{theorem}{Theorem}
\theoremstyle{theorem}\newtheorem{corollary}{Corollary}
\theoremstyle{theorem}\newtheorem{lemma}{Lemma}
\theoremstyle{theorem}\newtheorem{proposition}{Proposition}

\newcommand{\A}{\mathcal{A}}
\newcommand{\B}{\mathcal{B}}
\newcommand{\C}{\mathbb{C}}
\newcommand{\CC}{\mathcal{C}}
\newcommand{\D}{\mathcal{D}}
\renewcommand{\d}{\delta}
\newcommand{\E}{\mathcal{E}}
\newcommand{\e}{\varepsilon}
\newcommand{\F}{\mathbb{F}}
\newcommand{\FF}{\mathcal{F}}
\newcommand{\G}{\mathcal{G}}
\renewcommand{\H}{\mathbb{H}}
\newcommand{\I}{\mathcal{I}}
\newcommand{\J}{\mathcal{J}}
\newcommand{\K}{\mathbb{K}}
\renewcommand{\L}{\mathscr{L}}
\newcommand{\M}{\mathcal{M}}
\newcommand{\N}{\mathbb{N}}
\renewcommand{\O}{\mathcal{O}}
\renewcommand{\P}{\mathcal{P}}
\newcommand{\Q}{\mathbb{Q}}
\newcommand{\R}{\mathbb{R}}
\renewcommand{\S}{\mathcal{S}}
\newcommand{\T}{\mathbb{T}}
\newcommand{\U}{\mathcal{U}}
\newcommand{\V}{\mathcal{V}}
\newcommand{\W}{\mathcal{W}}
\newcommand{\X}{\mathcal{X}}
\newcommand{\Y}{\mathcal{Y}}
\newcommand{\Z}{\mathbb{Z}}

\begin{document}

\noindent \textbf{MATH 4161 Mathematics of Cryptography} \hfill \textbf{Assignment 8} \\
\textsc{Assignment} \hfill \textsc{Joe Tran}

\begin{question}
    Assume that you know that the word of the ciphertext \textsf{ZQ!G} corresponds to the plaintext \textsf{PICK}. Determine what (four letter English word) corresponds to the ciphertext: \textsf{KVLW}, and determine the key.
\end{question}

\begin{solution}
    Recall that we have the numerical representation of each letter given as follows:
    \begin{table}[h]
        \centering
        \resizebox{\columnwidth}{!}{%
        \begin{tabular}{ccccccccccccccccccccccccccccc}
        A            & B            & C            & D            & E            & F            & G            & H            & I            & J            & K            & L            & M            & N            & O            & P            & Q            & R            & S            & T            & U            & V            & W            & X            & Y            & Z            & .            & !            & ?            \\
        $\Downarrow$ & $\Downarrow$ & $\Downarrow$ & $\Downarrow$ & $\Downarrow$ & $\Downarrow$ & $\Downarrow$ & $\Downarrow$ & $\Downarrow$ & $\Downarrow$ & $\Downarrow$ & $\Downarrow$ & $\Downarrow$ & $\Downarrow$ & $\Downarrow$ & $\Downarrow$ & $\Downarrow$ & $\Downarrow$ & $\Downarrow$ & $\Downarrow$ & $\Downarrow$ & $\Downarrow$ & $\Downarrow$ & $\Downarrow$ & $\Downarrow$ & $\Downarrow$ & $\Downarrow$ & $\Downarrow$ & $\Downarrow$ \\
        0            & 1            & 2            & 3            & 4            & 5            & 6            & 7            & 8            & 9            & 10           & 11           & 12           & 13           & 14           & 15           & 16           & 17           & 18           & 19           & 20           & 21           & 22           & 23           & 24           & 25           & 26           & 27           & 28          
        \end{tabular}%
        }
    \end{table}
    
    We first seek a key of the form $k = \begin{bmatrix} a & b \\ c & d \end{bmatrix}$, where $a, b, c, d \in \Z_{29}$. Since the ciphertext ZQ!G corresponds to the numerical value 25-16-27-6 and the plaintext PICK corresponds to the numerical value 15-8-2-10, then we have that
    \begin{equation*}
        \begin{bmatrix} a & b \\ c & d \end{bmatrix} \begin{bmatrix} 15 \\ 8 \end{bmatrix} = \begin{bmatrix} 25 \\ 16 \end{bmatrix}
    \end{equation*}
    and
    \begin{equation*}
        \begin{bmatrix} a & b \\ c & d \end{bmatrix} \begin{bmatrix} 2 \\ 10 \end{bmatrix} = \begin{bmatrix} 27 \\ 6 \end{bmatrix}
    \end{equation*}
    Or in other words,
    \begin{equation*}
        \begin{bmatrix} a & b \\ c & d \end{bmatrix} \begin{bmatrix} 15 & 2 \\ 8 & 10 \end{bmatrix} = \begin{bmatrix} 25 & 27 \\ 16 & 6 \end{bmatrix}
    \end{equation*}
    So now right multiplying $\begin{bmatrix} 15 & 2 \\ 8 & 10 \end{bmatrix}^{-1}$ yields
    \begin{align*}
        \begin{bmatrix} a & b \\ c & d \end{bmatrix} &= \begin{bmatrix} 25 & 27 \\ 16 & 6 \end{bmatrix} \begin{bmatrix} 15 & 2 \\ 8 & 10 \end{bmatrix}^{-1} \\
        &= \begin{bmatrix} 25 & 27 \\ 16 & 6 \end{bmatrix} (134 \bmod 29)^{-1} \begin{bmatrix} 10 & -2 \\ -8 & 15 \end{bmatrix} \\
        &= \begin{bmatrix} 25 & 27 \\ 16 & 6 \end{bmatrix} 18^{-1} \begin{bmatrix} 10 & 27 \\ 21 & 15 \end{bmatrix}
    \end{align*}
    Now note that because $\gcd(18, 29) = 1$, then by Bezout's Theorem, there exists $x, y \in \Z$ such that
    \begin{equation*}
        18x + 29y = 1
    \end{equation*}
    In particular, $x = -8$ and $y = 5$ so that
    \begin{equation*}
        18(-8) + 29(5) = 1
    \end{equation*}
    so $-8 \equiv 21$ is the inverse of 18. Now,
    \begin{align*}
        \begin{bmatrix} a & b \\ c & d \end{bmatrix} &= \begin{bmatrix} 25 & 27 \\ 16 & 6 \end{bmatrix} 21 \begin{bmatrix} 10 & 27 \\ 21 & 15 \end{bmatrix} \\
        &= \begin{bmatrix} 25 & 27 \\ 16 & 6 \end{bmatrix} \begin{bmatrix} 7 & 16 \\ 6 & 25 \end{bmatrix} \\
        &= \begin{bmatrix} 18 & 2 \\ 3 & 0 \end{bmatrix}
    \end{align*}
    So now we check that the key that we have obtained is correct. Indeed,
    \begin{align*}
        \begin{bmatrix} 18 & 2 \\ 3 & 0 \end{bmatrix} \begin{bmatrix} 15 \\ 8 \end{bmatrix} = \begin{bmatrix} 25 \\ 16 \end{bmatrix}
    \end{align*}
    and also,
    \begin{equation*}
        \begin{bmatrix} 18 & 2 \\ 3 & 0 \end{bmatrix} \begin{bmatrix} 2 \\ 10 \end{bmatrix} = \begin{bmatrix} 27 \\ 6 \end{bmatrix}
    \end{equation*}
    as desired. So from here, our keyword is 18-2-3-0, which translates to SCDA. Finally, we now need to determine what KVLW is in plaintext, given our key. So, the ciphertext KVLW in numerical form is 10-21-11-22, so there exists $x, y, z, w \in \Z_{29}$ such that
    \begin{equation*}
        \begin{bmatrix} 18 & 2 \\ 3 & 0 \end{bmatrix} \begin{bmatrix} x_1 \\ x_2 \end{bmatrix} = \begin{bmatrix} 10 \\ 21 \end{bmatrix}
    \end{equation*}
    and
    \begin{equation*}
        \begin{bmatrix} 18 & 2 \\ 3 & 0 \end{bmatrix} \begin{bmatrix} y_1 \\ y_2 \end{bmatrix} = \begin{bmatrix} 11 \\ 22 \end{bmatrix}
    \end{equation*}
    In particular, $\begin{bmatrix} x_1 \\ x_2 \end{bmatrix} = \begin{bmatrix} 7 \\ 0 \end{bmatrix}$ and $\begin{bmatrix} y_1 \\ y_2 \end{bmatrix} = \begin{bmatrix} 22/3 \\ 53/2 \end{bmatrix}$. Now we just need to compute the values $22 \cdot 3^{-1} \bmod 29$ and $53 \cdot 2^{-1} \bmod 29$. Since $\gcd(3, 29) = 1$ and $\gcd(2, 29) = 1$, then by Bezout's Theorem, there exists $k, \ell, m, n \in \Z$ such that
    \begin{equation*}
        3k + 29\ell = 1
    \end{equation*}
    and
    \begin{equation*}
        2m + 29n = 1
    \end{equation*}
    In particular, $k = 10$, $\ell = -1$, $m = -14$ and $n = 1$. So 10 is the inverse of 3, and $-14 \equiv 15$ is the inverse of 2. Now,
    \begin{equation*}
        22 \cdot 3^{-1} \bmod 29 = 22 \cdot 10 \bmod 29 = 17
    \end{equation*}
    and
    \begin{equation*}
        53 \cdot 2^{-1} \bmod 29 = 53 \cdot 15 \bmod 29 = 12
    \end{equation*}
    So $\begin{bmatrix} y_1 \\ y_2 \end{bmatrix} = \begin{bmatrix} 17 \\ 12 \end{bmatrix}$. Note that now $\begin{bmatrix} x_1 \\ x_2 \end{bmatrix} = \begin{bmatrix} 7 \\ 0 \end{bmatrix} = \begin{bmatrix} H \\ A \end{bmatrix} $ and $\begin{bmatrix} y_1 \\ y_2 \end{bmatrix} = \begin{bmatrix} 17 \\ 12 \end{bmatrix} = \begin{bmatrix} R \\ M \end{bmatrix}$.

    To find the decrypting key, we need to find the inverse of the key. Indeed,
    \begin{align*}
        \begin{bmatrix} 18 & 2 \\ 3 & 0 \end{bmatrix}^{-1} &= (18 \cdot 0 - 3 \cdot 2)^{-1} \begin{bmatrix} 0 & -2 \\ -3 & 18 \end{bmatrix} \\
        &= (-6)^{-1} \begin{bmatrix} 0 & -2 \\ -3 & 18 \end{bmatrix} \\
        &= 23^{-1} \begin{bmatrix} 0 & -2 \\ -3 & 18 \end{bmatrix}
    \end{align*}
    Since $\gcd(23, 29) = 1$, then there exists $\d, \e \in \Z$ such that
    \begin{equation*}
        23\d + 29\e = 1
    \end{equation*}
    In particular, $\d = -5$ and $\e = 4$, so the inverse of 23 is $-5 \equiv 24$. So then,
    \begin{align*}
        \begin{bmatrix} 18 & 2 \\ 3 & 0 \end{bmatrix}^{-1} &= 24 \begin{bmatrix} 0 & -2 \\ -3 & 18 \end{bmatrix} \\
        &= \begin{bmatrix} 0 & 10 \\ 15 & 26 \end{bmatrix}
    \end{align*}
    To check that we have the deciphering key, observe that
    \begin{equation*}
        \begin{bmatrix} 0 & 10 \\ 15 & 26 \end{bmatrix}\begin{bmatrix} 10 \\ 21 \end{bmatrix} = \begin{bmatrix} 7 \\ 0 \end{bmatrix}
    \end{equation*}
    and
    \begin{equation*}
        \begin{bmatrix} 0 & 10 \\ 15 & 26 \end{bmatrix} \begin{bmatrix} 11 \\ 22 \end{bmatrix} = \begin{bmatrix} 17 \\ 12 \end{bmatrix}
    \end{equation*}
    as required.

    To conclude:
    \begin{itemize}
        \item Plaintext of KVLW: HARM
        \item Enciphering Key: $\begin{bmatrix} 18 & 2 \\ 3 & 0 \end{bmatrix}$
        \item Deciphering Key: $\begin{bmatrix} 0 & 10 \\ 15 & 26 \end{bmatrix}$
    \end{itemize}
\end{solution}

\end{document}