\documentclass[11pt]{amsart}
\usepackage[utf8]{inputenc}
\usepackage[margin=1in]{geometry}
\usepackage{amsfonts, amssymb, amsmath, amsthm, booktabs, hyperref, pgfplots, tikz, xcolor}

\theoremstyle{definition}\newtheorem{definition}{Definition}
\theoremstyle{definition}\newtheorem{question}{Question}
\theoremstyle{definition}\newtheorem*{solution}{Solution}

\date{January 9, 2024}

\setlength{\parskip}{5pt}

\begin{document}

\noindent \textbf{MATH 4161 Mathematics of Cryptography} \hfill \textbf{Assignment 1} \\
\noindent \textsc{Lecture} \hfill \textsc{Joe Tran}

\begin{question}
    Decipher the following ciphertext into plaintext: \textbf{\textsf{XQVHH QSDWK VDKHD G}} and determine the key.
\end{question}

\begin{solution}
    Starting with the ciphertext given by
    \begin{verbatim}
        X Q V H H  Q S D W K  V D K H D  G
        = = = = =  = = = = =  = = = = =  =
        Y R W I I  R T E X L  W E L I E  H
        Z S X J J  S U F Y M  X F M J F  I
        A T Y K K  T V G Z N  Y G N K G  J
        B U Z L L  U W H A O  Z H O L H  K
        C V A M M  V X I B P  A I P M I  L
        D W B N N  W Y J C Q  B J Q N J  M
        E X C O O  X Z K D R  C K R O K  N
        F Y D P P  Y A L E S  D L S P L  O
        G Z E Q Q  Z B M F T  E M T Q M  P
        H A F R R  A C N G U  F N U R N  Q
        I B G S S  B D O H V  G O V S O  R
        J C H T T  C E P I W  H P W T P  S
        K D I U U  D F Q J X  I Q X U Q  T
        L E J V V  E G R K Y  J R Y V R  U
        M F K W W  F H S L Z  K S Z W S  V
        N G L X X  G I T M A  L T A X T  W
        O H M Y Y  H J U N B  M U B Y U  X
        P I N Z Z  I K V O C  N V C Z V  Y
        Q J O A A  J L W P D  O W D A W  Z
        R K P B B  K M X Q E  P X E B X  A
        S L Q C C  L N Y R F  Q Y F C Y  B
        T M R D D  M O Z S G  R Z G D Z  C
        U N S E E  N P A T H  S A H E A  D <-
        V O T F F  O Q B U I  T B I F B  E
        W P U G G  P R C V J  U C J G C  F
    \end{verbatim}
    The plaintext is then \verb|UNSEE NPATH SAHEA D|
\end{solution}

For the key, observe that because from the plaintext U gets mapped to ciphertext X, then
\begin{center}
    \begin{tabular}{cccccccccccccccccccccccccc}
        A & B & C & D & E & F & G & H & I & J & K & L & M & N & O & P & Q & R & S & T & U & V & W & X & Y & Z \\
        $\Uparrow$ & $\Uparrow$ & $\Uparrow$ & $\Uparrow$ & $\Uparrow$ & $\Uparrow$ & $\Uparrow$ & $\Uparrow$ & $\Uparrow$ & $\Uparrow$ & $\Uparrow$ & $\Uparrow$ & $\Uparrow$ & $\Uparrow$ & $\Uparrow$ & $\Uparrow$ & $\Uparrow$ & $\Uparrow$ & $\Uparrow$ & $\Uparrow$ & $\Uparrow$ & $\Uparrow$ & $\Uparrow$ & $\Uparrow$ & $\Uparrow$ & $\Uparrow$ \\
        X & Y & Z & A & B & C & D & E & F & G & H & I & J & K & L & M & N & O & P & Q & R & S & T & U & V & W
    \end{tabular}
\end{center}
So our key is D.

\end{document}