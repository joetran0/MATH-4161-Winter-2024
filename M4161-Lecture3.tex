\documentclass[11pt]{article}
\usepackage[utf8]{inputenc}
\usepackage[margin=1in]{geometry}
\usepackage{amsfonts, amssymb, amsmath, amsthm, booktabs, hyperref, pgfplots, tikz, xcolor}

\theoremstyle{definition}\newtheorem{definition}{Definition}
\theoremstyle{definition}\newtheorem{example}{Example}
\theoremstyle{definition}\newtheorem{samplecode}{Sample Code}

\begin{document}

\noindent \textbf{MATH 4161 Mathematics of Cryptography} \hfill \textbf{Lecture 3} \\
\noindent \textsc{Lecture} \hfill \textsc{Joe Tran}

\tableofcontents

\section{Rectangular Transposition Keys}

A rectangular transposition key is the number of columns. For example, 4132576 is a permutation. Alternatively, use a key word \textsf{BATHROOM}, in which case,
\begin{center}
    \begin{tabular}{cccccccc}
        B & A & T & H & R & O & O & M \\
        2 & 1 & 8 & 3 & 7 & 5 & 6 & 4
    \end{tabular}
\end{center}

\section{Playfair}

Key:
\begin{center}
    \begin{tabular}{|c|c|c|c|c|} \hline
        D & E & N & I & A \\ \hline
        L & B & C & F & G \\ \hline
        H & K & M & O & P \\ \hline
        Q & R & S & T & U \\ \hline
        V & W & X & Y & Z \\ \hline
    \end{tabular}
\end{center}
Plaintext:
\begin{center}
    \textsf{382 Robertson Dr, West Hollywood}
\end{center}
Whenever we see numbers in the plaintext, we convert them into their word, i.e. 3 = three. Some rules:
\begin{enumerate}
    \item Stick with A-Z
    \item Replace J with I
    \item Convert any numbers to text
    \item Break into 2-grams and no adjacent letters equal. Insert a Q to make sure that no 2-gram has equal letters.
\end{enumerate}
From the above,
\begin{center}
    \textsf{TH RE EQ EI GH TQ TW OR OB ER TS ON DR WE ST HO LQ LY WO OD}
\end{center}
Then we obtain, in order,
\begin{center}
    \textsf{qo wb dr na lp ur ry kt kf bw ut mi eq eb tu kp hv fv yk hi}
\end{center}
Some other rules to keep in mind:
\begin{enumerate}
    \item If two letters in the key are in the same row, then move to the right for the encrypted 2-gram. For example, take B and F. Because they are in the same row, then move to the right to obtain the encrypted 2-gram cg.
    \item If two letters in the key are in the same column, then move down for the encrypted 2-gram. For example, take B and R. Because they are in the same row, then move down to obtain the encrypted 2-gram kw. 
    \item If two letters are in opposite corners,
    \begin{enumerate}
        \item Ex. Considering RF in the 2-gram. Move the R to the T and move the F to the B to obtain the encrypted 2-gram TB.
        \item Ex. Considering BT in the 2-gram. Move the B to the F and move the T to the R to obtain the encrypted 2-gram FR.
    \end{enumerate}
    \item If two letters are in the same row, but a letter goes outside the box, then wrap it around. For example, consider RN in the 2-gram. Then shifting to the right gives PL because the N moves outside the box, so we wrap it back to the first letter of the row, which is L.
    \item If two letters are in the same column, but a letter goes outside the box, then wrap it around. For example, consider BW in the 2-gram. Then moving down gives KE because W moves outside the box, so we wrap it back to the first letter of the column, which is E.
\end{enumerate}

\begin{example}
    Given the phrase: \textsf{Office floor plans} and given the plaintext \textsf{Massive heart attack}, then our key is
    \begin{center}
        \begin{tabular}{|c|c|c|c|c|} \hline
            O & F & I & C & E \\ \hline
            L & R & P & A & N \\ \hline
            S & B & D & G & H \\ \hline
            K & M & Q & T & U \\ \hline
            V & W & X & Y & Z \\ \hline
        \end{tabular}
    \end{center}
    Then splitting the plaintext into two grams:
    \begin{center} \small
        \begin{tabular}{|cc|cc|cc|cc|cc|cc|cc|cc|cc|cc|} \hline
            M & A & S & Q & S & I & V & E & H & E & A & R & T & A & T & Q & T & A & C & K \\ \hline 
            T & R & D & K & D & O & Z & O & V & N & N & P & Y & G & U & T & Y & G & O & T \\ \hline
        \end{tabular}
    \end{center}
    So the ciphertext is \textsf{tr dk do zo vn np yg ut yg ot}.
\end{example}

For playfair, the key must be a $5 \times 5$ box.

\section{Homophonic Substitution}

This one is called the homophonic substitution, which is similar to the Caesar substitution.
\begin{center}\tiny
    \begin{tabular}{c|ccccccccccccccccccccccccc}
        & A & B & C & D & E & F & G & H & I/J & K & L & M & N & O & P & Q & R & S & T & U & V & W & X & Y & Z \\ \hline
        S & 9 & 10 & 11 & 12 & 13 & 14 & 15 & 16 & 17 & 18 & 19 & 20 & 21 & 22 & 23 & 24 & 25 & 1 & 2 & 3 & 4 & 5 & 6 & 7 & 8 \\
        T & 8 & 9 & 10 & 11 & 12 & 13 & 14 & 15 & 16 & 17 & 18 & 19 & 20 & 21 & 22 & 23 & 24 & 25 & 1 & 2 & 3 & 4 & 5 & 6 & 7 \\
        A & 1 & 2 & 3 & 4 & 5 & 6 & 7 & 8 & 9 & 10 & 11 & 12 & 13 & 14 & 15 & 16 & 17 & 18 & 19 & 20 & 21 & 22 & 23 & 24 & 25 \\
        N & 14 & 15 & 16 & 17 & 18 & 19 & 20 & 21 & 22 & 23 & 24 & 25 & 1 & 2 & 3 & 4 & 5 & 6 & 7 & 8 & 9 & 10 & 11 & 12 & 13 \\
    \end{tabular}
\end{center}

This one is called a polyalphabet substitution.
\begin{center}\tiny
    \begin{tabular}{c|ccccccccccccccccccccccccc}
        & A & B & C & D & E & F & G & H & I/J & K & L & M & N & O & P & Q & R & S & T & U & V & W & X & Y & Z \\ \hline
        S & 9 & 10 & 11 & 12 & 13 & 14 & 15 & 16 & 17 & 18 & 19 & 20 & 21 & 22 & 23 & 24 & 25 & 1 & 2 & 3 & 4 & 5 & 6 & 7 & 8 \\
        T & 33 & 34 & 35 & 36 & 37 & 38 & 39 & 40 & 41 & 42 & 43 & 44 & 45 & 46 & 47 & 48 & 49 & 50 & 26 & 27 & 28 & 29 & 30 & 31 & 32 \\
        A & 51 & 52 & 53 & 54 & 55 & 56 & 57 & 58 & 59 & 60 & 61 & 62 & 63 & 64 & 65 & 66 & 67 & 68 & 69 & 70 & 71 & 72 & 73 & 74 & 75 \\
        N & 89 & 90 & 91 & 92 & 93 & 94 & 95 & 96 & 97 & 98 & 99 & 100 & 76 & 77 & 78 & 79 & 80 & 81 & 82 & 83 & 84 & 85 & 86 & 87 & 88 \\
    \end{tabular}
\end{center}
Here, our key is \textsf{STAN}. Given the plaintext \textsf{Then why did you turn some of us inside out?}, we can convert this using the table above.
\begin{verbatim}
T  H  E  N   W  H Y D  I D Y O  U T U R  N S O M  E O F U  S I N S  I D E O  U T ?
82 16 55 45  85 16 ...
\end{verbatim}
Or, given the ciphertext \textsf{69-16-9-2-85-33-81-35-51-25-61-40-13-1-45-93-85-20-64-77} reads that was carlhesnewmoo (??)

\end{document}