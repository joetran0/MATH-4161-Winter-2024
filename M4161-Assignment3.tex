\documentclass[11pt]{amsart}
\usepackage[utf8]{inputenc}
\usepackage[margin=1in]{geometry}
\usepackage{amsfonts, amssymb, amsmath, amsthm, booktabs, hyperref, pgfplots, tikz, xcolor}

\theoremstyle{definition}\newtheorem{definition}{Definition}
\theoremstyle{definition}\newtheorem{question}{Question}
\theoremstyle{definition}\newtheorem*{solution}{Solution}

\date{January 9, 2024}

\setlength{\parskip}{5pt}

\begin{document}

\noindent \textbf{MATH 4161 Mathematics of Cryptography} \hfill \textbf{Assignment 3} \\
\noindent \textsc{Lecture} \hfill \textsc{Joe Tran}

\begin{question}
    Given the ciphertext \textsf{EEAIR GGITS GYUEE TDNUW NOAOY RIAAT RHWGR NDVLP CITHN ITAOF KSGIO EATTD OELLB FHTDN AI}, and the length of the keyword is 6, use rectangular transposition to find the plaintext and the key. The plaintext contains the word \textsf{plastic}.
\end{question}

\begin{solution}
    Because the length of the ciphertext is 72, and the length of the keyword is 6, then
    \begin{equation*}
        k = \frac{\text{length of ciphertext}}{\text{length of keyword}} = \frac{72}{6} = 12
    \end{equation*}
    So we make the ciphertext into 12-grams.
    \begin{center}
        \textsf{EEAIRGGITSGY UEETDNUWNOAO YRIAATRHWGRN DVLPCITHNITA OFKSGIOEATTD OELLBFHTDNAI}
    \end{center}
    So we have the rectangle given as follows:
    \begin{equation*}
        \begin{bmatrix}
            E & U & Y & D & O & O \\
            E & E & R & V & F & E \\
            A & E & I & L & K & L \\
            I & T & A & P & S & L \\
            R & D & A & C & G & B \\
            G & N & T & I & I & F \\
            G & U & R & T & O & H \\
            I & W & H & H & E & T \\
            T & N & W & N & A & D \\
            S & O & G & I & T & N \\
            G & A & R & T & T & A \\
            Y & O & N & A & D & I \\
        \end{bmatrix}
    \end{equation*}
    Because the plaintext contains the word plastic, then we need to rearrange the columns until the word plastic appears.
    \begin{equation*}
        \begin{bmatrix}
            E & U & Y & D & O & O \\
            E & E & R & V & F & E \\
            A & E & I & L & K & L \\
            I & T & A & P & S & L \\
            R & D & A & C & G & B \\
            G & N & T & I & I & F \\
            G & U & R & T & O & H \\
            I & W & H & H & E & T \\
            T & N & W & N & A & D \\
            S & O & G & I & T & N \\
            G & A & R & T & T & A \\
            Y & O & N & A & D & I \\
        \end{bmatrix} \Rightarrow \begin{bmatrix}
            D & O & Y & O & U & E \\
            V & E & R & F & E & E \\
            L & L & I & K & E & A \\
            P & L & A & S & T & I \\
            C & B & A & G & D & R \\
            I & F & T & I & N & G \\
            T & H & R & O & U & G \\
            H & T & H & E & W & I \\
            N & D & W & A & N & T \\
            I & N & G & T & O & S \\
            T & A & R & T & A & G \\
            A & I & N & D & O & Y \\
        \end{bmatrix}
    \end{equation*}
    Therefore, the plaintext is: \textsf{Do you feel like a plastic bag drifting through the wind wanting to start again} with extra letters being \textsf{doy}.

    We now need a permutation that satisfies the order in which the rectangular transposition was put in. Observe that the order of the permutation has the form
    \begin{equation*}
        \begin{pmatrix}
            1 & 2 & 3 & 4 & 5 & 6 \\
            6 & 5 & 3 & 1 & 4 & 2
        \end{pmatrix}
    \end{equation*}
    So the key is 463521, i.e. $(123456) \mapsto (463521)$.
\end{solution}

\end{document}