\documentclass[11pt]{article}
\usepackage[utf8]{inputenc}
\usepackage[margin=1in]{geometry}
\usepackage{amsfonts, amssymb, amsmath, amsthm, booktabs, hyperref, pgfplots, tikz, xcolor}

\theoremstyle{definition}\newtheorem{definition}{Definition}
\theoremstyle{definition}\newtheorem{question}{Question}
\theoremstyle{definition}\newtheorem*{solution}{Solution}

\date{January 9, 2024}

\setlength{\parskip}{5pt}

\begin{document}

\noindent \textbf{MATH 4161 Mathematics of Cryptography} \hfill \textbf{Assignment 4} \\
\noindent \textsc{Assignment} \hfill \textsc{Joe Tran}

\begin{question}
    Given the ciphertext \textsf{RB BL RP ST AP VY GY XC TX CA ZA GO BE YD QG IC CF ZM WL WH DK PL ST UW DK RC OQ AW} and given the partial key
    \begin{center}
        \begin{tabular}{|c|c|c|c|c|} \hline
            O & X & F & V & M \\ \hline
            Y & K & W & * & N \\ \hline
            * & T & R & L & * \\ \hline
            Q & B & * & A & * \\ \hline
            S & I & P & D & G \\ \hline
        \end{tabular}
    \end{center}
    where * means that there is a letter that you have to fill in. Determine the plaintext and determine the key.
\end{question}

\begin{solution}
    First note that the key is missing the following letters: C, E, H, U, and Z. Using the above table, we can find part of the plaintext as follows:
    \begin{center}
        \begin{tabular}{|cc|cc|cc|cc|cc|cc|cc|cc|cc|cc|} \hline
            R & B & B & L & R & P & S & T & A & P & V & Y & G & Y & X & C & T & X & C & A \\ \hline
            T & * & A & T & W & * & I & * & * & D & O & * & S & N & • & • & K & I & • & • \\ \hline
        \end{tabular}

        \begin{tabular}{|cc|cc|cc|cc|cc|cc|cc|cc|cc|cc|} \hline
            Z & A & G & O & B & E & Y & D & Q & G & I & C & C & F & Z & M & W & L & W & H \\ \hline
            • & • & S & M & • & • & * & S & * & S & • & • & • & • & • & • & * & R & • & • \\ \hline
        \end{tabular}

        \begin{tabular}{|cc|cc|cc|cc|cc|cc|cc|cc|cc|cc|} \hline
            D & K & P & L & S & T & U & W & D & K & R & C & O & Q & A & W \\ \hline
            I & * & D & R & I & * & • & • & I & * & • & • & S & * & * & * \\ \hline
        \end{tabular}
    \end{center}
    where • denotes the letter that cannot be determined at the moment.

    First observe the first four letters T*AT. One of the letters that can be used in this case is H, and so the fourth row then becomes
    \begin{center}
        \begin{tabular}{|c|c|c|c|c|} \hline
            Q & B & \textcolor{red}{H} & A & * \\ \hline
        \end{tabular}
    \end{center}
    making the first four letters THAT. Then filling out what we have,
    \begin{center}
        \begin{tabular}{|cc|cc|cc|cc|cc|cc|cc|cc|cc|cc|} \hline
            R & B & B & L & R & P & S & T & A & P & V & Y & G & Y & X & C & T & X & C & A \\ \hline
            T & \textcolor{red}{H} & A & T & W & \textcolor{red}{H} & I & * & \textcolor{red}{H} & D & O & * & S & N & • & • & K & I & • & • \\ \hline
        \end{tabular}

        \begin{tabular}{|cc|cc|cc|cc|cc|cc|cc|cc|cc|cc|} \hline
            Z & A & G & O & B & E & Y & D & Q & G & I & C & C & F & Z & M & W & L & W & H \\ \hline
            • & • & S & M & • & • & * & S & * & S & • & • & • & • & • & • & * & R & F & R \\ \hline
        \end{tabular}

        \begin{tabular}{|cc|cc|cc|cc|cc|cc|cc|cc|cc|cc|} \hline
            D & K & P & L & S & T & U & W & D & K & R & C & O & Q & A & W \\ \hline
            I & * & D & R & I & * & • & • & I & * & • & • & S & * & \textcolor{red}{H} & * \\ \hline
        \end{tabular}
    \end{center}

    Next, observe the next five letters after THAT, WHI*H. One of the letters that can be used in this case is C, and so the third row becomes
    \begin{center}
        \begin{tabular}{|c|c|c|c|c|} \hline
            \textcolor{blue}{C} & T & R & L & * \\ \hline
        \end{tabular}
    \end{center}
    making the five letters WHICH. Then filling out what we have,
    \begin{center}
        \begin{tabular}{|cc|cc|cc|cc|cc|cc|cc|cc|cc|cc|} \hline
            R & B & B & L & R & P & S & T & A & P & V & Y & G & Y & X & C & T & X & C & A \\ \hline
            T & \textcolor{red}{H} & A & T & W & \textcolor{red}{H} & I & \textcolor{blue}{C} & \textcolor{red}{H} & D & O & * & S & N & O & T & K & I & L & Q \\ \hline
        \end{tabular}

        \begin{tabular}{|cc|cc|cc|cc|cc|cc|cc|cc|cc|cc|} \hline
            Z & A & G & O & B & E & Y & D & Q & G & I & C & C & F & Z & M & W & L & W & H \\ \hline
            • & • & S & M & • & • & * & S & * & S & S & T & R & O & • & • & * & R & F & R \\ \hline
        \end{tabular}

        \begin{tabular}{|cc|cc|cc|cc|cc|cc|cc|cc|cc|cc|} \hline
            D & K & P & L & S & T & U & W & D & K & R & C & O & Q & A & W \\ \hline
            I & * & D & R & I & \textcolor{blue}{C} & • & • & I & * & • & • & S & \textcolor{blue}{C} & \textcolor{red}{H} & * \\ \hline
        \end{tabular}
    \end{center}

    Next, observe the two 3-grams WH, DK and PL, FRI*DR. Ignoring the last R for the moment, one of the letters that we can use in this situation is E, and so making the second row
    \begin{center}
        \begin{tabular}{|c|c|c|c|c|} \hline
            Y & K & W & \textcolor{green}{E} & N \\ \hline
        \end{tabular}
    \end{center}
    and thus, we have FRIEDR. Then filling what we can,
    \begin{center}
        \begin{tabular}{|cc|cc|cc|cc|cc|cc|cc|cc|cc|cc|} \hline
            R & B & B & L & R & P & S & T & A & P & V & Y & G & Y & X & C & T & X & C & A \\ \hline
            T & \textcolor{red}{H} & A & T & W & \textcolor{red}{H} & I & \textcolor{blue}{C} & \textcolor{red}{H} & D & O & \textcolor{green}{E} & S & N & O & T & K & I & L & Q \\ \hline
        \end{tabular}

        \begin{tabular}{|cc|cc|cc|cc|cc|cc|cc|cc|cc|cc|} \hline
            Z & A & G & O & B & E & Y & D & Q & G & I & C & C & F & Z & M & W & L & W & H \\ \hline
            • & • & S & M & A & K & \textcolor{green}{E} & S & * & S & S & T & R & O & • & • & \textcolor{green}{E} & R & F & R \\ \hline
        \end{tabular}

        \begin{tabular}{|cc|cc|cc|cc|cc|cc|cc|cc|cc|cc|} \hline
            D & K & P & L & S & T & U & W & D & K & R & C & O & Q & A & W \\ \hline
            I & \textcolor{green}{E} & D & R & I & \textcolor{blue}{C} & • & • & I & \textcolor{green}{E} & T & * & S & \textcolor{blue}{C} & \textcolor{red}{H} & \textcolor{green}{E} \\ \hline
        \end{tabular}
    \end{center}

    Now, observe the 2-gram QG. One of the letters that can be used in this case is U, so the fourth row becomes
    \begin{center}
        \begin{tabular}{|c|c|c|c|c|} \hline
            Q & B & \textcolor{red}{H} & A & \textcolor{magenta}{U} \\ \hline
        \end{tabular}
    \end{center}
    making the 2-gram US in plaintext. So now,
    \begin{center}
        \begin{tabular}{|cc|cc|cc|cc|cc|cc|cc|cc|cc|cc|} \hline
            R & B & B & L & R & P & S & T & A & P & V & Y & G & Y & X & C & T & X & C & A \\ \hline
            T & \textcolor{red}{H} & A & T & W & \textcolor{red}{H} & I & \textcolor{blue}{C} & \textcolor{red}{H} & D & O & \textcolor{green}{E} & S & N & O & T & K & I & L & Q \\ \hline
        \end{tabular}

        \begin{tabular}{|cc|cc|cc|cc|cc|cc|cc|cc|cc|cc|} \hline
            Z & A & G & O & B & E & Y & D & Q & G & I & C & C & F & Z & M & W & L & W & H \\ \hline
            * & \textcolor{magenta}{U} & S & M & A & K & \textcolor{green}{E} & S & \textcolor{magenta}{U} & S & S & T & R & O & • & • & \textcolor{green}{E} & R & F & R \\ \hline
        \end{tabular}

        \begin{tabular}{|cc|cc|cc|cc|cc|cc|cc|cc|cc|cc|} \hline
            D & K & P & L & S & T & U & W & D & K & R & C & O & Q & A & W \\ \hline
            I & \textcolor{green}{E} & D & R & I & \textcolor{blue}{C} & H & N & I & \textcolor{green}{E} & T & * & S & \textcolor{blue}{C} & \textcolor{red}{H} & \textcolor{green}{E} \\ \hline
        \end{tabular}
    \end{center}

    Lastly, it follows that the last remaining letter is Z, which belongs to the third row
    \begin{center}
        \begin{tabular}{|c|c|c|c|c|} \hline
            \textcolor{blue}{C} & T & R & L & \textcolor{cyan}{Z} \\ \hline
        \end{tabular}
    \end{center}
    So now,
    \begin{center}
        \begin{tabular}{|cc|cc|cc|cc|cc|cc|cc|cc|cc|cc|} \hline
            R & B & B & L & R & P & S & T & A & P & V & Y & G & Y & X & C & T & X & C & A \\ \hline
            T & \textcolor{red}{H} & A & T & W & \textcolor{red}{H} & I & \textcolor{blue}{C} & \textcolor{red}{H} & D & O & \textcolor{green}{E} & S & N & O & T & K & I & L & Q \\ \hline
        \end{tabular}

        \begin{tabular}{|cc|cc|cc|cc|cc|cc|cc|cc|cc|cc|} \hline
            Z & A & G & O & B & E & Y & D & Q & G & I & C & C & F & Z & M & W & L & W & H \\ \hline
            L & \textcolor{magenta}{U} & S & M & A & K & \textcolor{green}{E} & S & \textcolor{magenta}{U} & S & S & T & R & O & N & G & \textcolor{green}{E} & R & F & R \\ \hline
        \end{tabular}

        \begin{tabular}{|cc|cc|cc|cc|cc|cc|cc|cc|cc|cc|} \hline
            D & K & P & L & S & T & U & W & D & K & R & C & O & Q & A & W \\ \hline
            I & \textcolor{green}{E} & D & R & I & \textcolor{blue}{C} & H & N & I & \textcolor{green}{E} & T & \textcolor{cyan}{Z} & S & \textcolor{blue}{C} & \textcolor{red}{H} & \textcolor{green}{E} \\ \hline
        \end{tabular}
    \end{center}

    Therefore, the plaintext is \textsf{That which does not kill us makes us stronger Friedrich Nietzsche} and the key is the table given by
    \begin{center}
        \begin{tabular}{|c|c|c|c|c|} \hline
            O & X & F & V & M \\ \hline
            Y & K & W & \textcolor{green}{E} & N \\ \hline
            \textcolor{blue}{C} & T & R & L & \textcolor{cyan}{Z} \\ \hline
            Q & B & \textcolor{red}{H} & A & \textcolor{magenta}{U} \\ \hline
            S & I & P & D & G \\ \hline
        \end{tabular}
    \end{center}
\end{solution}

\end{document}